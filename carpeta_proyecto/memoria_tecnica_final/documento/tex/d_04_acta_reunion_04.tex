% D.04. ACTA DE REUNION #04
%----------------------------------------------------------------------------------------------	
\begin{center}	
\Large{Acta de Reunión Nº04\hspace{0.25em}-\hspace{0.25em}\tituloReunion}
\end{center}
\vspace{1.5em}

% PRIMERA TABLA: INFORMACIÓN BÁSICA
\begin{longtable}{ | L{\dimexpr 0.420\linewidth-2\tabcolsep} |
				     L{\dimexpr 0.570\linewidth-2\tabcolsep} | }
\hline % ------------------------------------------------------------------------
\rowcolor{grisCabeceraTabla}
\mc{2}{\bf Información básica}  \\
%\hline % ------------------------------------------------------------------------
%{\bf Cliente} & Nombre del cliente (por definir o no es necesario ?)  \\
\hline % ------------------------------------------------------------------------
{\bf Proyecto} & $\mu$Search \\
\hline % ------------------------------------------------------------------------
{\bf Fecha y hora de reunión} & 06/03/14 - 10:00 \\
\hline % ------------------------------------------------------------------------
{\bf Lugar} & Seminario 25, Edif. Ada Byron. EINA. Unizar \\
\hline % ------------------------------------------------------------------------
{\bf Tipo de reunión} & Reunión TP6 con Profesor \\
\hline % ------------------------------------------------------------------------
\end{longtable}


%----------------------------------------------------------------------------------------
% SEGUNDA TABLA - ASISTENTES
\begin{longtable}{ | C{\dimexpr 0.070\linewidth-2\tabcolsep} |
                     L{\dimexpr 0.350\linewidth-2\tabcolsep} |
                     C{\dimexpr 0.370\linewidth-2\tabcolsep} |
                     C{\dimexpr 0.200\linewidth-2\tabcolsep} | }
\hline % ------------------------------------------------------------------------
\rowcolor{grisCabeceraTabla}
\mc{4}{\bf Asistentes} \\
\hline % ------------------------------------------------------------------------
{\bf Nº} & {\bf Nombre y Apellidos} & {\bf Cargo} & {\bf Rol} \\
\hline % ------------------------------------------------------------------------
{\bf 1} & Alberto Berbel Aznar & Verificación y validación & --  \\
\hline % ------------------------------------------------------------------------
{\bf 2} & Javier Briz Alastrué & Gestor de configuraciones & --  \\
\hline % ------------------------------------------------------------------------
{\bf 3} & Héctor Francia Molinero & Gestor de calidad & --  \\
\hline % ----------------------------------------------------
{\bf 4} & Daniel García Páez & Director del proyecto & -- \\
\hline % ------------------------------------------------------------------------
{\bf 5} & Alejandro Gracia Mateo & Gestor de planificación & --  \\
\hline % ------------------------------------------------------------------------
{\bf 6} & Simón Ortego Parra & Gestor de desarrollo & Secretario  \\
\hline % ------------------------------------------------------------------------
{\bf 7} & Fco. Javier Zaragaza Soria & -- & Preparador  \\
\hline % ------------------------------------------------------------------------
\end{longtable}

%----------------------------------------------------------------------------------------
% CUARTA TABLA - OBJETIVOS, CUERPO DE LA REUNIÓN, DECISIONES TOMADAS, ...
\begin{longtable}{ | C{\dimexpr\linewidth-2\tabcolsep} | }
\hline % ------------------------------------------------------------------------
\rowcolor{grisCabeceraTabla}
{\bf Objetivos} \\
\hline % ------------------------------------------------------------------------
\endfirsthead
\hline % ------------------------------------------------------------------------
\endhead
\espacioSubtabla
\hline % ------------------------------------------------------------------------
\endfoot
\hline % ------------------------------------------------------------------------
\endlastfoot

\itemNvlUno{El objetivo principal de está reunion es el de realizar un seguimiento, con el profesor preparador Fco. Javier Zaragza, de nuestra propuesta de proyecto y de como se está abordando la primera iteración del proyecto software a realizar para su lanzamiento para poder ir cerrando y concretando varios apartados.}\\

% ------------------------------------------------------------------------
\rowcolor{grisCabeceraTabla}
{\bf Cuerpo de la reunión} \\
\hline % ------------------------------------------------------------------------

\itemNvlUno{Cada uno de los asistentes se presentó a la hora.}
\itemNvlUno{El equipo pusó al día al preparador sobre el avance en la propuesta de proyecto, todo aquello que ya estaba decidido. }
\itemNvlUno{El preparador nos ofreció varios consejos y pautas para la modificación y mejora de los requisitos que habíamos establecido en reuniones previas.}
\itemNvlUno{Se pasó a discutir ampliamente con el preparador para recibir su consejo sobre dos de los aspectos más complicados de abordar de la propuesta de proyecto:}
\itemNvlDos{Planificación de la 1ª iteración del proyecto en tareas y asignación de horas aproximadas a cada tarea. Que diferentes técnicas se pueden seguir para realizar la aproximación.}
\itemNvlDos{Realización de la oferta de presupuesto del proyecto.}
\itemNvlUno{Por último, se negoció la fijación de fechas para:}
\itemNvlDos{Siguiente reunión de seguimiento de la primera iteración.}
\itemNvlDos{Lanzamiento de la segunda iteración.}
\\

\hline % ------------------------------------------------------------------------
\rowcolor{grisCabeceraTabla}
{\bf Decisiones tomadas}  \\
\hline % ------------------------------------------------------------------------

\itemNvlUno{Se decidió que había que mejorar y detallar ciertos aspectos de los requisitos del sistema:}
\itemNvlDos{La definición de elemento: microcontrolador caracterizado por nº de referencia, arquitectura, flash, memoria, ...}
\itemNvlDos{Establecer en base a que característica de los microcontroladores se mostrarán ordenados los listados de búsqueda.}
\itemNvlDos{Establecer los datos personales que el cliente deberá rellenar a la hora de pedir la generación de una factura.}
\itemNvlDos{En caso de existir alguna restricción o limitación sobre el sistema o plataforma de desarrollo del proyecto software, añadirlas como requisitos no funcionales.}
\itemNvlUno{Se decide que la división de tareas tanto para la planificación como para el presupuesto se realizará en base a los requisitos establecidos.}
\itemNvlUno{Se incluirá una primera actividad o tarea denominada como "Tarea 0" que será el lanzamiento del proyecto, incluyendo la estructura inicial del proyecto: zona de arranque, instalación del SGBD, creación de la BD...}
\itemNvlUno{Se decidió la no utilización de máquinas virtuales para trabajar en el mismo entorno. Cada miembro trabajará en su propio equipo con las versiones de las tecnologías (PHP, MySQL, HTML...) a utilizar ya establecidas.}
\itemNvlUno{Se decide modificar la hoja de esfuerzos proporcionada en clase, de forma que todas las tareas que cada miembro del equipo haga irán incluida en una de la siguientes tareas:}
\itemNvlDos{Tareas ligadas a los requisitos.}
\itemNvlDos{Documentación.}
\itemNvlDos{Pruebas.}
\itemNvlDos{Reuniones técnicas y no técnicas (o de planificación).}
\\

\hline % ------------------------------------------------------------------------
\rowcolor{grisCabeceraTabla}
{\bf Temas pendientes} \\
\hline % ------------------------------------------------------------------------

\itemNvlUno{Establecimiento de las fechas de:}
\itemNvlDos{Reunión de seguimiento de la primera iteración.}
\itemNvlDos{Fecha de lanzamiento de la segunda iteración.}
\itemNvlUno{Modificación final de los requisitos del sistema para la propuesta de proyecto.}
\itemNvlUno{Especificación de las tareas en las que se dividirá la planificación del proyecto y realización de la oferta de presupuesto a partir de las misma.}
\\

\hline % ------------------------------------------------------------------------
\rowcolor{grisCabeceraTabla}
{\bf Próxima reunión prevista} \\
\hline % ------------------------------------------------------------------------
Reunión semanal  12/03/2014 20:00
\end{longtable}

