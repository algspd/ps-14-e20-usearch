% 09.1. LECCIONES APRENDIDAS
%----------------------------------------------------------------------------------------------
Se enumeran aquí alguna de las lecciones aprendidas por los miembros del equipo durante el desarrollo del proyecto:
\begin{itemize}
	\item Se ha aprendido la importancia del reparto de roles. Si bien la dimensión del proyecto no era quizás lo suficientemente grande para sacarle el máximo beneficio a esto, sí que sin este tipo de organización hubiera sido mucho más díficil llevar todo a cabo. Teniendo cada miembro un papel bien definido dentro del proyecto es mucho más fácil el realizar todas las tareas y juntar el trabajo periódicamente.
	\item A pesar de tener varios canales de comunicación entre los miembros del equipo (\textit{Whatsapp}, correo electrónico (\textit{Google Groups})...), es muy importante las reuniones periódicas para poner en común el trabajo. Es decir, en persona y cara a cara es mucho más efectivo trabajar, es muy importante para el desarrollo del proyecto.
	\item Se ha aprendido la importancia de buscar siempre varias alternativas a los problemas que surgen, poner ideas en común y llegar a un acuerdo como grupo. Por ejemplo, la decisión final de realizar el proyecto implementando la el catálogo con un \textit{Framework} como \textit{CodeIgniter} fue una decisión consensuada y finalmente vital para realizar a tiempo el proyecto. Así como la decisión de utilizar \textit{LaTex} para la documentación, decisión consensuada también y que ha agilizado mucho todo el desarrollo del proyecto.
	\item Se ha aprendido la importancia de que los miembros del equipo no sólo se ciñan a su papel, sino que también estén al día como mínimo de otros aspectos del proyecto. Pues si algún día faltaba un miembro o no estaba disponible durante algún tiempo, alguien tenía que suplirle temporalmente. De igual manera, es importante que el director esté al tanto de todo y se interese por el trabajo del resto de miembros del equipo.
	\item Se ha aprendido la importancia de planificar un proyecto antes de comenzarlo. Sí aún planificándolo ha habido momentos de cierta descordinación, sin la planificación pudiera haber sido un desastre que se ha evitado.
	\item Se ha aprendido la importancia de utilizar una herramienta de control de versiones como \textit{SubVersion} para desarrollo del proyecto, pudiendo deshacer cambios y errores, teniendo al instante los cambios realizados por otro compañero, etc. Además de aprender a utilizar este tipo de herramientas para futuros proyectos.
\end{itemize}
\paragraph{}En resumen, estás son algunas de las lecciones más importantes aprendidas, pero ha sido de gran importancia el aprender en general como trabajar en grupo con roles definidos y planificando cada aspecto de un proyecto.