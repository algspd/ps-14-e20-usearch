% D.08. ACTA DE REUNION #08
%----------------------------------------------------------------------------------------------
\begin{center}	
\Large{Acta de Reunión Nº8\hspace{0.25em}-\hspace{0.25em}\tituloReunion}
\end{center}
\vspace{1.5em}

% PRIMERA TABLA: INFORMACIÓN BÁSICA
\begin{longtable}{ | L{\dimexpr 0.420\linewidth-2\tabcolsep} |
				     L{\dimexpr 0.570\linewidth-2\tabcolsep} | }
\hline % ------------------------------------------------------------------------
\rowcolor{grisCabeceraTabla}
\mc{2}{\bf Información básica}  \\
%\hline % ------------------------------------------------------------------------
%{\bf Cliente} & Nombre del cliente (por definir o no es necesario ?)  \\
\hline % ------------------------------------------------------------------------
{\bf Proyecto} & $\mu$Search \\ 
\hline % ------------------------------------------------------------------------
{\bf Fecha y hora de comienzo} & 27/03/14 - 10:00 \\
\hline % ------------------------------------------------------------------------
{\bf Lugar} & Seminario 21, Edif. Ada Byron. EINA. Unizar \\
\hline % ------------------------------------------------------------------------
{\bf Tipo de reunión} & Estándar \\
\hline % ------------------------------------------------------------------------
\end{longtable}


%----------------------------------------------------------------------------------------
% SEGUNDA TABLA - ASISTENTES
\begin{longtable}{ | C{\dimexpr 0.070\linewidth-2\tabcolsep} |
                     L{\dimexpr 0.350\linewidth-2\tabcolsep} |
                     C{\dimexpr 0.370\linewidth-2\tabcolsep} |
                     C{\dimexpr 0.200\linewidth-2\tabcolsep} | }
\hline % ------------------------------------------------------------------------
\rowcolor{grisCabeceraTabla}
\mc{4}{\bf Asistentes} \\
\hline % ------------------------------------------------------------------------
{\bf Nº} & {\bf Nombre y Apellidos} & {\bf Cargo} & {\bf Rol} \\
\hline % ------------------------------------------------------------------------
{\bf 1} & Alberto Berbel Aznar & Verificación y validación & --  \\
%\hline % ------------------------------------------------------------------------
%{\bf 2} & Javier Briz Alastrué & Gestor de configuraciones & --  \\
\hline % ------------------------------------------------------------------------
{\bf 3} & Héctor Francia Molinero & Gestor de calidad & --  \\
\hline % ----------------------------------------------------
{\bf 4} & Daniel García Páez & Director del proyecto & -- \\
\hline % ------------------------------------------------------------------------
{\bf 5} & Alejandro Gracia Mateo & Gestor de planificación & --  \\
\hline % ------------------------------------------------------------------------
{\bf 6} & Simón Ortego Parra & Gestor de desarrollo & Secretario  \\
\hline % ------------------------------------------------------------------------
\end{longtable}


%----------------------------------------------------------------------------------------
% TERCERA TABLA - AUSENTES
\begin{longtable}{ | C{\dimexpr 0.070\linewidth-2\tabcolsep} |
                     L{\dimexpr 0.350\linewidth-2\tabcolsep}  |
                     C{\dimexpr 0.570\linewidth-2\tabcolsep} | }
\hline % ------------------------------------------------------------------------
\rowcolor{grisCabeceraTabla}
\mc{3}{\bf Ausentes} \\ 
\hline % ------------------------------------------------------------------------
{\bf 1} & Javier Briz Alastrué & Gestor de configuraciones \\
\hline % ------------------------------------------------------------------------
\end{longtable}


%----------------------------------------------------------------------------------------
% CUARTA TABLA - OBJETIVOS, CUERPO DE LA REUNIÓN, DECISIONES TOMADAS, ...
\begin{longtable}{ | C{\dimexpr\linewidth-2\tabcolsep} | }
\hline % ------------------------------------------------------------------------
\rowcolor{grisCabeceraTabla}
{\bf Cuerpo de la reunión} \\
\hline % ------------------------------------------------------------------------
\endfirsthead
\hline % ------------------------------------------------------------------------
\endhead
\espacioSubtabla
\hline % ------------------------------------------------------------------------
\endfoot
\hline % ------------------------------------------------------------------------
\endlastfoot
\itemNvlUno{Cada uno de los asistentes se presentó a la hora.}

\itemNvlUno{El profesor asistente realizó una serie de preguntas
a los integrantes del equipo:}

	\itemNvlDos{En primer lugar, preguntó sobre el número de horas que
	habían sido dedicadas al proyecto y si se había realizado una estimación
	del porcentaje del proyecto que se llevaba desarrollado. La respuesta
	del equipo fue que no se había realizado dicha estimación, pero que
	aparecía en las hojas de esfuerzos individuales el número de horas
	dedicadas. El profesor también aconsejo que se realizase esta
	estimación no todas las semanas, pero sí siempre antes de una
	reunión de éste carácter.}

	\itemNvlDos{La segunda pregunta que planteó fue si había ocurrido
	o se estaban tratando con alguna dificultad en el desarrollo. El equipo
	respondió negativamente, y así se finalizó una parte importante de la
	reunión que estaba planteada para resolver cualquier problema que 
	pudiera haber ocurrido.}

\itemNvlUno{Los miembros del equipo le plantearon la fecha de entrega
de la primera iteración del proyecto, a lo que el profesor añadió que
era importante no olvidar realizar una estimación apropiada de las pruebas
y fijarlas en el calendario, teniendo en cuenta también que en la segunda
iteración solía ocurrir que había menos tiempo semanalmente para la 
dedicación al proyecto. A continuación, el profesor aconsejó realizar
una plantilla (estándar) para la realización de las pruebas y que era
un buen consejo que no realizara las pruebas la misma persona que 
había realizado la parte del código pertinente.}

\itemNvlUno{Como último aspecto importante el profesor recordó
que había que realizar las auditorías (una externa, realizada
por otro grupo, además de otra interna, como mínimo) y realizar
la planificación y el ajuste del calendario en base a éstas.}

\itemNvlUno{Por último se planteó una fecha para la siguiente reunión,
que podría cambiar si le venía mal a alguno de los convocados.}\\

\hline % ------------------------------------------------------------------------
\rowcolor{grisCabeceraTabla}
{\bf Próxima reunión prevista} \\
\hline % ------------------------------------------------------------------------
Viernes 09 de Mayo del 2014 a las 12:00 \\
\end{longtable}
