% D.11. ACTA DE REUNION #11
%----------------------------------------------------------------------------------------------	
\begin{center}	
\Large{Acta de Reunión Nº11\hspace{0.25em}-\hspace{0.25em}\tituloReunion}
\end{center}
\vspace{1.5em}

% PRIMERA TABLA: INFORMACIÓN BÁSICA
\begin{longtable}{ | L{\dimexpr 0.420\linewidth-2\tabcolsep} |
				     L{\dimexpr 0.570\linewidth-2\tabcolsep} | }
\hline % ------------------------------------------------------------------------
\rowcolor{grisCabeceraTabla}
\mc{2}{\bf Información básica}  \\
%\hline % ------------------------------------------------------------------------
%{\bf Cliente} & Nombre del cliente (por definir o no es necesario ?)  \\
\hline % ------------------------------------------------------------------------
{\bf Proyecto} & $\mu$Search \\
\hline % ------------------------------------------------------------------------
{\bf Fecha y hora de reunión} & 09/04/14 - 20:00 \\
\hline % ------------------------------------------------------------------------
{\bf Lugar} & Seminario 21, Edif. Ada Byron. EINA. Unizar \\
\hline % ------------------------------------------------------------------------
{\bf Tipo de reunión} & Estándar \\
\hline % ------------------------------------------------------------------------
\end{longtable}


%----------------------------------------------------------------------------------------
% SEGUNDA TABLA - ASISTENTES
\begin{longtable}{ | C{\dimexpr 0.070\linewidth-2\tabcolsep} |
                     L{\dimexpr 0.350\linewidth-2\tabcolsep} |
                     C{\dimexpr 0.370\linewidth-2\tabcolsep} |
                     C{\dimexpr 0.200\linewidth-2\tabcolsep} | }
\hline % ------------------------------------------------------------------------
\rowcolor{grisCabeceraTabla}
\mc{4}{\bf Convocados} \\
\hline % ------------------------------------------------------------------------
{\bf Nº} & {\bf Nombre y Apellidos} & {\bf Cargo} & {\bf Rol} \\
\hline % ------------------------------------------------------------------------
{\bf 1} & Alberto Berbel Aznar & Verificación y validación & --  \\
\hline % ------------------------------------------------------------------------
{\bf 2} & Javier Briz Alastrué & Gestor de configuraciones & Cronometrador  \\
\hline % ------------------------------------------------------------------------
{\bf 3} & Héctor Francia Molinero & Gestor de calidad & Observador  \\
\hline % ----------------------------------------------------
{\bf 4} & Daniel García Páez & Director del proyecto & Director \\
\hline % ------------------------------------------------------------------------
{\bf 5} & Alejandro Gracia Mateo & Gestor de planificación & Planificador \\
\hline % ------------------------------------------------------------------------
{\bf 6} & Simón Ortego Parra & Gestor de desarrollo & Secretario \\
\hline % ------------------------------------------------------------------------
\end{longtable}

%----------------------------------------------------------------------------------------
% CUARTA TABLA - OBJETIVOS, CUERPO DE LA REUNIÓN, DECISIONES TOMADAS, ...
\begin{longtable}{ | C{\dimexpr\linewidth-2\tabcolsep} | }
\hline % ------------------------------------------------------------------------
\rowcolor{grisCabeceraTabla}\textbf{Objetivo de la reunión} \\
\hline % ------------------------------------------------------------------------
\endfirsthead
\hline % ------------------------------------------------------------------------
\endhead
\espacioSubtabla
\hline % ------------------------------------------------------------------------
\endfoot
\hline % ------------------------------------------------------------------------
\endlastfoot

\itemNvlUno{Reunión para comprobar que todas las tareas del proyecto asignadas a la primera iteración del mismo han sido finalizadas y comenzar la planificación de la segunda iteración del proyecto.} \\

\hline % ------------------------------------------------------------------------
\rowcolor{grisCabeceraTabla}\textbf{Cuerpo de la reunión} \\
\hline % ------------------------------------------------------------------------

\itemNvlUno{Cada uno de los asistentes se presentó a la hora.}
\itemNvlUno{Se realizó una revisión de todas las tareas asiganadas a la primera iteración del proyecto para cada uno de los miembros del equipo.
Todas y cada una de ellas estaba finalizada, por lo tanto se dió por cerrada la primera iteración en cuanto a tareas a realizar.}
\itemNvlUno{Se realizó un revisión de las hojas de esfuerzo de todos los miembros del equipo, asi como del recuento total de horas de trabajo del equipo, comparándolas con la planificación realizada en la propuesta de proyecto y poder asi observar la desviación en horas de trabajo sobre dicha planificación inicial.
Pensando cara a la auditoría externa, se creó una nueva hoja de Excel que recoge las horas utilizadas y porcentajes sobre la estimación inicial realizada, representando los resultados con gráficas.}
\itemNvlUno{Por último, se realizó una planificación inicial para las tareas de la segunda iteración del proyecto, asignándolas entre los componentes del equipo de la manera más repartida posible y teniendo en cuenta las horas de la primera iteración recinetemente revisadas, con el fin de compensar las desviaciones existentes de horas de trabajo.}
\\

\hline % ------------------------------------------------------------------------
\rowcolor{grisCabeceraTabla}\textbf{Decisiones tomadas} \\
\hline % ------------------------------------------------------------------------
\itemNvlUno{Se da por cerrada la primera iteración del proyecto.}
\itemNvlUno{Se crea una hoja Excel para represenar mediante tablas y gráficas los porccentajes de horas de trabajo cumplidos de la primera iterción del proyecot, tanto individuales como globales.}
\itemNvlUno{Se crea un archivo de texto en el directoria SVN del proyecto con la asignación incial de tareas para la segunda iteración, y por lo tanto se pone en marcha la segunda iteración del proyecto.}
\\
% ------------------------------------------------------------------------
\rowcolor{grisCabeceraTabla}\textbf{Temas pendientes} \\
\hline % ------------------------------------------------------------------------

\itemNvlUno{Refinar los resultados de la hoja de Excel.}
\itemNvlUno{Quedan pendientes realizar todas las tareas asignadas para el la segunda iteración del proyecto.}\\
	
\hline % ------------------------------------------------------------------------
\rowcolor{grisCabeceraTabla}\textbf{Próxima reunión prevista} \\
\hline % ------------------------------------------------------------------------
Viernes 21 de Mayo del 2014 a las 20:00 \\
\end{longtable}

