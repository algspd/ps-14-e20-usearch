\documentclass[10pt,spanish]{article}

%----------------------------------------------------------------------------------------
% Codificación, y usar una fuente similar a la Palatino (y no Latin Modern Roman)
\usepackage[utf8]{inputenc} % Acentos, etc.
\usepackage[T1]{fontenc}
\usepackage[spanish]{babel} % Castellano
\usepackage{tgpagella}      % Fuente similar a Palatino

%----------------------------------------------------------------------------------------
% MÁRGENES (menores que en otro tipo de documentos)
\addtolength{\oddsidemargin}{-.375in}
\addtolength{\evensidemargin}{-.375in}
\addtolength{\textwidth}{1.25in}

%\addtolength{\topmargin}{-.375in}
%\addtolength{\textheight}{.5in}

%----------------------------------------------------------------------------------------
% COLORES
\usepackage{xcolor,colortbl} % Colores personalizados y color fondo tablas
\definecolor{grisCabeceraTabla}{RGB}{220,220,220}
\definecolor{grisHeader}{RGB}{180,180,180}
	
%----------------------------------------------------------------------------------------
% TABLAS	  
%\usepackage{supertabular} % Tablas multipágina
\linespread {1.4} % Tamaño que ocupa una línea (celda)

\usepackage{array}
\usepackage{longtable}
\setlength\LTleft{0pt}
\setlength\LTright{0pt}
\setlength{\columnsep}{0em}

% Definir los estilos de las celdas para indentar el texto
\newcolumntype{L}[1]{>{\raggedright\let\newline\\\arraybackslash\hspace{0pt}}m{#1}}
\newcolumntype{C}[1]{>{\centering\let\newline\\\arraybackslash\hspace{0pt}}m{#1}}
\newcolumntype{R}[1]{>{\raggedleft\let\newline\\\arraybackslash\hspace{0pt}}m{#1}}


% Filas con columnas multiples
\newcommand{\mc}[2]{\multicolumn{#1}{|C{\dimexpr 1\linewidth-2\tabcolsep}|}{#2}}

% Colores de lineas de las tablas
%\arrayrulecolor{grisCabeceraTabla}
\arrayrulecolor{grisHeader}

% Eliminar espacio previo en lista de ítems
\usepackage{enumitem}
\setlist{nolistsep}

% Comandos especiales para crear items dentro de tabla y poder hacer
% una tabla de multiples paginas y que se pueda realizar el salto de
% pagina en cualquier punto.
\usepackage{scrextend}
\newenvironment{lvlOneItem}
	{\begin{addmargin}[2.5em]{1em}
	\vspace{-1em}
	\hspace{-1em}$\bullet$\hspace{0.5em}}
	{\vspace{-2em}\end{addmargin}}
\newenvironment{lvlTwoItem}
	{\begin{addmargin}[4em]{1em}
	\vspace{-1em}
	\hspace{-1em}$\circ$\hspace{0.5em}}
	{\vspace{-2em}\end{addmargin}}

	
\newcommand{\itemNvlUno}[1]{\begin{lvlOneItem}{#1}\end{lvlOneItem}\\}
\newcommand{\itemNvlDos}[1]{\begin{lvlTwoItem}{#1}\end{lvlTwoItem}\\}
\newcommand{\espacioSubtabla}{\\[0.5ex]}
\newcommand{\cabeceraTabla}[1]{\rowcolor{grisCabeceraTabla}{\bf #1}}

%----------------------------------------------------------------------------------------
% METADATOS DEL PDF Y PDF CLICKEABLE
\usepackage{hyperref}
\usepackage{hyperxmp}

% DATOS A CAMBIAR
\newcommand{\numeroDeReunion}{02}
\newcommand{\tituloReunion}{\bf Lanzamiento del proyecto}
\newcommand{\nombreDelProyecto}{$\mu$Search}

\hypersetup{
	pdfauthor={Alberto Berbel Aznar, 
				Javier Briz Alastrué, 
				Héctor Francia Molinero, 
				Daniel García Páez,
				Alejandro Gracia Mateo,
				Simón Ortego Parra},
	pdftitle={E20 - Acta R\numeroDeReunion: \tituloReunion},
	pdfsubject={Proyecto Software. Grado Ing. Informática. EINA. Unizar},
	pdfkeywords={},
	pdfcopyright={Copyright (C) 2014 by Alberto Berbel Aznar, 
				Javier Briz Alastrué, 
				Héctor Francia Molinero, 
				Daniel García Páez,
				Alejandro Gracia Mateo,
				Simón Ortego Parra. All rights reserved.},
	pdfproducer={PDFLatex},
	pdfcreator={ps2pdf},
	colorlinks=false
}

%----------------------------------------------------------------------------------------
% Encabezados y pies de pagina : FancyHdr
\usepackage{fancyhdr}
\pagestyle{fancy}
\fancyhf{} % borrar todos los ajustes
\setlength{\headheight}{15pt}
\usepackage{lastpage} % Para poner total de paginas en el footer, ej: pag 1/4

\fancyhead[L]{\color{grisHeader}{\large BITPARTY}}
\fancyhead[R]{\color{grisHeader}E20 - Convocatoria R\numeroDeReunion \\ \nombreDelProyecto}
\fancyfoot[R]{\color{grisHeader}\thepage/\pageref*{LastPage}}

% Modifica el ancho de las líneas de cabecera y pie
\renewcommand{\headrulewidth}{0pt}
\renewcommand{\footrulewidth}{0pt}
\renewcommand{\headsep}{0.6in}
\renewcommand{\headwidth}{6in}

%----------------------------------------------------------------------------------------
%----------------------------------------------------------------------------------------
% INICIO DEL DOCUMENTO
%----------------------------------------------------------------------------------------

\begin{document}
	
\begin{center}	
\Large{Convocatoria de Reunión Nº \numeroDeReunion\hspace{0.25em}-\hspace{0.25em}\tituloReunion}
\end{center}
\vspace{1.5em}

% PRIMERA TABLA: INFORMACIÓN BÁSICA
\begin{longtable}{ | L{\dimexpr 0.420\linewidth-2\tabcolsep} |
				     L{\dimexpr 0.570\linewidth-2\tabcolsep} | }
\hline % ------------------------------------------------------------------------
\rowcolor{grisCabeceraTabla}
\mc{2}{\bf Información básica}  \\
%\hline % ------------------------------------------------------------------------
%{\bf Cliente} & Nombre del cliente (por definir o no es necesario ?)  \\
\hline % ------------------------------------------------------------------------
{\bf Proyecto} & $\mu$Search \\
\hline % ------------------------------------------------------------------------
{\bf Fecha y hora de reunión} & 20/02/14 - 12:00 \\
\hline % ------------------------------------------------------------------------
{\bf Lugar} & Seminario 21, Edif. Ada Byron. EINA. Unizar \\
\hline % ------------------------------------------------------------------------
{\bf Tipo de reunión} & Estándar \\
\hline % ------------------------------------------------------------------------
\end{longtable}


%----------------------------------------------------------------------------------------
% SEGUNDA TABLA - ASISTENTES
\begin{longtable}{ | C{\dimexpr 0.070\linewidth-2\tabcolsep} |
                     L{\dimexpr 0.350\linewidth-2\tabcolsep} |
                     C{\dimexpr 0.370\linewidth-2\tabcolsep} |
                     C{\dimexpr 0.200\linewidth-2\tabcolsep} | }
\hline % ------------------------------------------------------------------------
\rowcolor{grisCabeceraTabla}
\mc{4}{\bf Convocados} \\
\hline % ------------------------------------------------------------------------
{\bf Nº} & {\bf Nombre y Apellidos} & {\bf Cargo} & {\bf Rol} \\
\hline % ------------------------------------------------------------------------
{\bf 1} & Alberto Berbel Aznar & Verificación y validación & --  \\
\hline % ------------------------------------------------------------------------
{\bf 2} & Javier Briz Alastrué & Gestor de configuraciones & --  \\
\hline % ------------------------------------------------------------------------
{\bf 3} & Héctor Francia Molinero & Gestor de calidad & --  \\
\hline % ----------------------------------------------------
{\bf 4} & Daniel García Páez & Director del proyecto & -- \\
\hline % ------------------------------------------------------------------------
{\bf 5} & Alejandro Gracia Mateo & Gestor de planificación & --  \\
\hline % ------------------------------------------------------------------------
{\bf 6} & Simón Ortego Parra & Gestor de desarrollo & Secretario  \\
\hline % ------------------------------------------------------------------------
\end{longtable}

%----------------------------------------------------------------------------------------
% CUARTA TABLA - PREPARACIÓN PREVIA Y AGENDA 
\begin{longtable}{ | C{\dimexpr\linewidth-2\tabcolsep} | }
\hline % ------------------------------------------------------------------------
\cabeceraTabla{Preparación previa} \\
\hline % ------------------------------------------------------------------------
\endfirsthead
\hline % ------------------------------------------------------------------------
\endhead
\espacioSubtabla
\hline % ------------------------------------------------------------------------
\endfoot

\itemNvlUno{Revisar documentación relacionada con el producto a realizar.}
\itemNvlUno{Revisar documentación de roles.}
\itemNvlUno{Revisar plantilla de propuesta de proyecto.} \\

\hline % ------------------------------------------------------------------------
\cabeceraTabla{Agenda} \\
\hline % ------------------------------------------------------------------------

\itemNvlUno{[2 min.] - Revisión de la agenda.}

\itemNvlUno{[10 min.] - Asignación de roles de proyecto (en caso de que no
estén ya asignados).}
	\itemNvlDos{Revisión de la documentación de los roles.}
	\itemNvlDos{Discusión y asignación de roles en base a capacidades de
	los miembros (tener en cuenta las fichas de información personal
	de los miembros del equipo).}
	\itemNvlDos{Rol de gestor de planificación.}
	\itemNvlDos{Rol de gestor de configuraciones.}
	\itemNvlDos{Rol de gestor de calidad.}
	\itemNvlDos{Rol de verificación y validación.}
	
\itemNvlUno{[25 min.] - Discusión y acuerdo general sobre los requisitos a 
cumplir para el proyecto (que posteriormente se detallarán en la propuesta del
proyecto).}
	\itemNvlDos{Plantear la lista de requisitos.}
	\itemNvlDos{Revisar someramente funcionales especiales.}
	
\itemNvlUno{[30 min.] - Planteamiento de la estrategia de diseño.}
	\itemNvlDos{Proposición y evaluación de estrategias técnicas alternativas.}
	\itemNvlDos{Planteamiento del diseño arquitectural del sistema.}
	\itemNvlDos{Diseño conceptual de componentes, establecer como subdividir
	y posteriormente integrar el producto.}
		
\itemNvlUno{[5 min.] - Planteamiento/elección de tecnologías y estándar/es.}
	\itemNvlDos{Codificación.}
	\itemNvlDos{...}
	
\itemNvlUno{[30 min.] - Propuesta tentativa de la planificación de alto nivel.}
	\itemNvlDos{Plantear potenciales riesgos importantes.}
	\itemNvlDos{Realizar estimaciones preliminares de esfuerzos.}
	\itemNvlDos{Repartir tentativamente la lista de requisitos en los 
	distintos ciclos.}
	\itemNvlDos{Proponer las pautas del presupuesto.}
	
\itemNvlUno{[15 min.] - Organización del trabajo para crear el documento de 
propuesta de proyecto.}
	\itemNvlDos{Elección del responsable de la propuesta de proyecto.}
	\itemNvlDos{Desglose de tareas a realizar.}
	\itemNvlDos{Estimación del esfuerzo.}
	\itemNvlDos{Reparto de tareas entre el equipo.}
	\itemNvlDos{Calendario de trabajo.}
	
\itemNvlUno{[3 min.] - \emph{Debrieffing}.}\\
\end{longtable}


\end{document}