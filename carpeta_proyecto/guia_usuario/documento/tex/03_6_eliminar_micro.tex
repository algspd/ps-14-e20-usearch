
\begin{itemize}
\item \textbf{Arquitectura}
\item \textbf{Frecuencia(MHz)}
\item \textbf{Flash(Kb})
\item \textbf{RAM(Kb}) 
\item \textbf{Precio($\euro$)}
\end{itemize}

\paragraph{} El campo \textbf{''Referencia''} no será editable.

\paragraph{} Al pulsar el botón \textbf{\textit{''Modificar''}} se modifica el elemento en la base de datos y automáticamente se redirigirá al administrador a la página de listado completo, con el nuevo microcontrolador ya incluido en la lista.

\paragraph{}Como en todas las demás páginas del sitio web, si se pulsa en cualquiera de los dos logotipos del catálogo $\mu$Search (esquina superior izquierda), el sistema redirigirá al administrador a la página inicial de administración del catálogo.

\paragraph{}Además, desde esta página de edición, a través de los iconos situados en la cabecera debajo de los logotipos de la web, el administrador puede acceder a:

\begin{itemize}
	\item \textbf{Crear Nuevo:} Pulsando sobre este botón el administrador es redirigido a la página que le permitirá añadir un nuevo microcontrolador a la base de datos del catálogo electrónico.

	\item \textbf{Búsqueda:} Desde esta sección de la cabecera, el administrador puede realizar búsquedas sobre el catálogo de microcontroladores en base a cualquiera de las diferentes características de un microcontrolador (Arquitectura, Frecuencia, Flash, RAM). Simplemente se debe seleccionar una de las características de la lista despegable, introducir el texto a buscar y pulsar sobre el icono de búsqueda.
	El administrador será redirigido a una página donde se le mostrará el resultado de la búsqueda en forma de lista de microcontroladores.
			
	\item \textbf{Listado Completo:} Pulsando sobre este botón/icono el sistema redirige al administrador a la página en la que se listan todos los elementos disponibles en el catálogo de microcontroladores, con sus correspondientes características, y desde donde podrán editar o eliminar.
\end{itemize}