Como en todas las demás páginas del sitio web, si se pulsa en cualquiera de los dos logotipos del catálogo uSearch,
el sistema redirigirá a la página inicial del catálogo.

Debajo de los dos logotipos anteriormente citados se encuentran dos iconos:
\begin{itemize}
	\item[Carrito de la compra] A su lado, aparece también el número de artículos 
	que han sido introducidos en el mismo, pulsando sobre este icono se vuelve a cargar la página actual. Volver a 				
  cargar la página actual puede ser útil en el caso de que se haya modificado la cantidad de uno o varios artículos 			
  del	carrito y se quiera deshacer estos cambios sin que tengan ningún efecto en el precio final de la compra.

	\item[LISTADO COMPLETO] Pulsando sobre este icono el sistema vuelve a la página en la que se listan todos los
	elementos disponibles en el catálogo de microcontroladores.
\end{itemize}

A continuación, se encuentra el listado de los microcontroladores que el usuario ha ido añadiendo al carrito de la 
compra. Este listado, esta compuesto por cuatro columnas: 
\begin{itemize}
	\item[Cantidad] La cantidad de cada elemento añadido al carrito de la compra es un campo modificable que permite
	solicitar más o menos unidades de ese elemento. Para que estos cambios tengan efecto es necesario actualizar el 
	listado como se explica en el apartado siguiente (botón Actualizar).
	
	\item[Referencia] Es el código del fabricante que identifica a cada elemento del catálogo, cada microcontrolador
	tiene una referencia única distinta de las de los demás.
	
	\item[Precio] Es el precio que tiene una unidad.

	\item[Subtotal] Es el resultado de multiplicar el número de unidades solicitadas de un elemento por el precio que
	tiene cada unidad. Hay tantos subtotales como elementos en el carrito de la compra.
\end{itemize}

Debajo del listado de los elementos del carrito de la compra, se encuentran dos botones que pueden tener efecto sobre
todos los microcontroladores solicitados:
\begin{itemize}
	\item[Actualizar] Este botón hay que pulsarlo después de modificar una o varias cantidades de los elementos del 
	carrito de la compra. Es el que permite que dichos cambios tengan efecto. Se puede observar cómo cambia el precio
	Total de la compra. Además, si alguna nueva cantidad tiene un valor de 0, el elemento que tenga dicha cantidad 			
  será eliminado del carrito de la compra.
	
	\item[Vaciar] Pulsando sobre este botón se vacía completamente el carrito de la compra.
\end{itemize}

Antes de finalizar el pedido, hay disponible un formulario que el cliente debe rellenar con sus datos de contacto 
para poder generar correctamente la factura asociada al pedido que se quiere realizar. Todos los campos son obligatorios.

Finamente, una vez que el cliente ha revisado que todos los datos relativos al pedido (cantidades, referencias, precios, 
datos personales, etc.) son correctos hay que pulsar sobre el botón Realizar Pedido para que el sistema genere la 
factura asociada a dicho pedido.