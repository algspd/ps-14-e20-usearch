\documentclass[letterpaper,12pt]{article}

\usepackage{currvita}         % Cargamos el módulo para Curriculum Vitae
\usepackage[spanish]{babel}   % Definimos el idioma español
\usepackage[latin1]{inputenc} % Cargamos el modulo de codificacion de caracteres para que acepte tildes y eñes
\usepackage{fullpage}         % Disminuye los márgenes, cabe más en cada página

\title{Curriculum Vitae}
\author{Héctor Francia Molinero }
%\date{\today}

\topmargin  -1cm   % Para reducir el margen superior e inferior
\textheight 26cm  % Alto hoja tamaño carta 27.81cm

\begin{document}

\setlength{\cvlabelwidth}{40mm}  % Modifica el espacio para las etiquetas de los listados

\begin{cv}{Curriculum Vitae}


Nombre completo: Héctor Francia Molinero


\begin{cvlist}{Lenguajes de programación utilizados}
\item Ada, Java, C, C++, Android
\item Ensamblador, ARM, ARM thumb
\item CLIPS, Haskell, erlang
\item HTML, CSS, XML
\item Matlab, SQL
\end{cvlist}

\begin{cvlist}{Experiencia}

	\item[2011] \textbf{Participación en la creación de una empresa de ocio.}\\
	
	\item[2011] \textbf{Creación de un programa para verificar documentos bien formados en XML con Flex y 			Bison}\\

	\item[2012] \textbf{Creación de un juego de dominó con programación concurrente.}\\
	
	\item[2013] \textbf{Realización de un compresor de ficheros de texto.}\\
	
	\item[2013] \textbf{Creación de una BD ``policial'' en MySQL.}\\
	
	\item[2013] \textbf{Realización de una aplicación móvil de gestión de notas para Android.}\\

	\item[2013] \textbf{Realización de una página de recomendación de películas.}\\
	
	\item[2013] \textbf{Realización de un sistema de chat para múltiples usuarios con Erlang y Java.}
	
	\item[2013] \textbf{Realización de un compilador para un lenguaje similar a MiniLang.}
	
	\item[2013] \textbf{Creación de una página Web usando el CMS Wordpress.}

	
	


\end{cvlist}

\begin{cvlist}{Formación}

	\item[2008 a 2011] Estudiante de \textbf{Ingeniería Superior informática}\\
		
	\item[2011 a 2014] Estudiante de la EINA
		Grado en \textbf{Ingeniería informática} en la rama de \textbf{computación}\\


\end{cvlist}

\end{cv}

\end{document}
