Se propone una solución basada en tecnologías web, capaces de resolver tanto los requisitos de interacción de la aplicación con el usuario como los problemas relacionados con tratamiento y persistencia interna de la información.
Concretamente, se utilizará una interfaz web compatible con las últimas versiones de los navegadores más utilizados (más adelante se detallará esto), y se utilizará el framework CodeIgniter, en lenguaje PHP, como base del proyecto. Para el almacenamiento de la información se utilizará una base de datos MySQL.
Para evitar problemas de latencias con la base de datos, y dado el reducido tamaño del sistema, se optará por alojar la base de datos y todo el resto del sistema (servidor web e intérprete PHP) en un mismo servidor.

Las tecnologías, lenguajes y aplicaciones propuestas para el desarrollo del proyecto son:

\begin{itemize}
\item HTML 5
\item CSS 3
\item PHP 5
\item CodeIgniter 2.1.4
\item MySQL 5.5
\end{itemize}

Se asegurará que la web renderice de forma correcta en los siguientes navegadores:

\begin{itemize}
\item Google Chrome >=30
\item Internet Explorer >=10
\item Mozilla Firefox >=27
\item Opera >=12
\end{itemize}

La documentación y manuales de usuario se entregarán al cliente en PDF.

