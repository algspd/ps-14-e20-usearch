\paragraph{} El proyecto se divide en tres fases principales:
\begin{enumerate}
	\item Lanzamiento del proyecto.
	\item Primera iteración del proyecto.
	\item Segunda iteración del proyecto.
\end{enumerate}

\paragraph{} Se desarrollan a continuación las actividades o tareas que componen cada una de las fases:
\begin{enumerate}
	\item \textbf{Lanzamiento del proyecto.}
		\begin{itemize}
		\item Proceso de obtención de los requisitos funcionales y no funcionales de la aplicación.
		\item Diseño de un prototipo sin funcionalidad de la interfaz web de la aplicación.
		\item Diseño de la base de datos con la que trabajará la aplicación.
		\item Instalación  y puesta a punto del sistema sobre el que funcionará la aplicación.
		\item Planificación del datagrama de actividades para las dos iteraciones del proyecto y realización de un presupuesto.
		\end{itemize}
	\item \textbf{Primera iteración del proyecto.}
		\begin{itemize}
		\item Población de la base de datos.
		\item TAREA 1: Implementación básica de la interfaz web de la aplicación.
		\item TAREA 2: Implementación del control para insertar, modificar y eliminar elementos en el catálogo.
		\item TAREA 3: Implementación del control para mostrar listados de elementos del catálogo.
		\item TAREA 4: Implementación del control para el carrito de compra de la aplicación.
		\item TAREA 5: Implementación del control para generar, en texto plano, pedidos de compra.
		\item Documentación técnica de todo lo implementado (documentación de codigo y mantenimiento de una \textit{Wiki}.)
		\item Documentación del manual y guía de usuario de lo implementado hasta el momento.
		\item Preparación y realización de pruebas.
		\end{itemize}
	\item \textbf{Segunda iteración el proyecto.}
		\begin{itemize}
		\item Población de la base de datos.
		\item TAREA 6: Implementación del control para realizar búsquedas en el catálogo.
		\item TAREA 7: Implementación del control para generar pedidos de compra en formato PDF.
		\item TAREA 8: Mejora de la implementación inicial de la interfaz web de la aplicación
		\item Documentación técnica de todo lo implementado (documentación de codigo y mantenimiento de una \textit{Wiki}.)
		\item Documentación del manual y guía de usuario final.
		\item Preparación y realización de pruebas.
		\item Cierre y puesta a punto final del proyecto.
		\end{itemize}
\end{enumerate}
\paragraph{} Se puede observar una clara relación entre los requisitos del sistema definidos anteriormente y las tareas ''TAREA X'' aquí definidas. Pues es de esa forma como se ha abordado la planificación de tareas y actividades.
\paragraph{}Además, las tareas de documentación y pruebas son comunes a ambas iteraciones y se realizan paralelamente a la implementación con la intención de adelantar trabajo y no olvidar cosas por el camino.