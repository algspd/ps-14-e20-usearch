\documentclass[11pt,spanish]{article}

%----------------------------------------------------------------------------------------
% Codificación, y usar una fuente similar a la Palatino (y no Latin Modern Roman)
\usepackage[utf8]{inputenc} % Acentos, etc.
\usepackage[spanish]{babel} % Castellano
\usepackage{tgpagella}      % Fuente similar a Palatino
\usepackage[T1]{fontenc}

%----------------------------------------------------------------------------------------
% TODO
\usepackage{color}
\newcommand{\TODO}[1]{\textcolor{red}{\textbf{\MakeUppercase{#1}}}}

%----------------------------------------------------------------------------------------
% MODIFICAR EL ESTILO DE LAS SECCIONES 
\usepackage{titlesec}
\titleformat{\section}{\bfseries\Large}{\thesection}{0.8em}{}
\titleformat{\subsection}{\bfseries\large}{\thesubsection}{0.8em}{}
\titleformat{\subsubsection}{\bfseries\large}{\thesubsubsection}{0.8em}{}
%\newcommand{\sectionbreak}{\clearpage} % Empezar secciones en nueva página

%----------------------------------------------------------------------------------------
% Write keywords with line breaks
\newcommand{\code}[1]{\texttt{\justify #1}}

%----------------------------------------------------------------------------------------
% Footnotes without line
\makeatletter
\newcommand\footnoteref[1]{\protected@xdef\@thefnmark{\ref{#1}}\@footnotemark}
\makeatother

%----------------------------------------------------------------------------------------
% Imagenes: Que se pongan donde yo quiero
\usepackage{graphicx}
%\usepackage{float}
\usepackage[linecolor=colComment,linewidth=0.65pt]{mdframed}

\DeclareGraphicsExtensions{.eps,.ps,.pdf,.png,.jpg,.jpeg}
\addto{\captionsspanish}{\renewcommand{\listfigurename}{\bfseries\Large Figuras}}

%----------------------------------------------------------------------------------------
% Continuar con numeracion arabiga 
\renewcommand{\thepage}{\roman{page}}

%----------------------------------------------------------------------------------------
% Definir el titulo

% Comando para multiples saltos de linea
\newcommand{\singlelinebreak}{\\[\baselineskip]}
\newcommand{\multiplelinebreak}[1]{\\[#1\baselineskip]}
\newlength{\drop}

\newcommand*{\titulo}{\begingroup
\thispagestyle{empty}
\drop = 0.13\textheight
\centering
\vfill
\vspace*{\drop}
{\Huge\bf BITPARTY}\multiplelinebreak{2}
{\huge\tt $\mu$Search}\multiplelinebreak{2}
{\Huge Propuesta}\singlelinebreak
{\Huge del}\singlelinebreak
{\Huge Proyecto}\multiplelinebreak{2}
\includegraphics[scale=0.4]{img/imgNotAvailable.jpeg}
\vfill
\vspace*{\drop}
\endgroup}


%----------------------------------------------------------------------------------------
% METADATOS DEL PDF Y PDF CLICKEABLE
\usepackage{hyperref}
\usepackage{hyperxmp}

% DATOS A CAMBIAR
\newcommand{\nombreDelProyecto}{$\mu$Search}

\hypersetup{
	pdfauthor={Alberto Berbel Aznar, 
				Javier Briz Alastrué, 
				Héctor Francia Molinero, 
				Daniel García Páez,
				Alejandro Gracia Mateo,
				Simón Ortego Parra},
	pdftitle={E20 - Propuesta del proyecto},
	pdfsubject={Proyecto Software. Grado Ing. Informática. EINA. Unizar},
	pdfkeywords={},
	pdfcopyright={Copyright (C) 2014 by Alberto Berbel Aznar, 
				Javier Briz Alastrué, 
				Héctor Francia Molinero, 
				Daniel García Páez,
				Alejandro Gracia Mateo,
				Simón Ortego Parra. All rights reserved.},
	pdfproducer={PDFLatex},
	pdfcreator={ps2pdf},
	colorlinks=false
}

\usepackage[capitalise]{cleveref} % load after hyperref package
\crefdefaultlabelformat{\textbf{#2#1#3}} % boldface only the number

\crefname{figure}{figura}{figuras}
\Crefname{figure}{Figura}{Figuras}

\crefname{listing}{algoritmo}{algoritmos}
\Crefname{listing}{Algoritmo}{Algoritmos}

\crefname{section}{sección}{secciones}
\Crefname{section}{Sección}{Secciones}

%----------------------------------------------------------------------------------------
%----------------------------------------------------------------------------------------
% INICIO DEL DOCUMENTO
%----------------------------------------------------------------------------------------


%----------------------------------------------------------------------------------------
% PORTADA: TÍTULO
%----------------------------------------------------------------------------------------
\begin{document}
\titulo
\clearpage


%----------------------------------------------------------------------------------------
% ABSTRACT: RESUMEN EJECUTIVO
%----------------------------------------------------------------------------------------
\thispagestyle{empty}
\null\vspace{\fill}
\begin{abstract}
Esto es el resumen ejecutivo. Esto es el resumen ejecutivo.
Esto es el resumen ejecutivo. Esto es el resumen ejecutivo.
Esto es el resumen ejecutivo. Esto es el resumen ejecutivo.
Esto es el resumen ejecutivo. Esto es el resumen ejecutivo.
Esto es el resumen ejecutivo. Esto es el resumen ejecutivo.
Esto es el resumen ejecutivo. Esto es el resumen ejecutivo.
Esto es el resumen ejecutivo. Esto es el resumen ejecutivo.
Esto es el resumen ejecutivo. Esto es el resumen ejecutivo.
Esto es el resumen ejecutivo. Esto es el resumen ejecutivo.

\end{abstract}

\vspace{\fill}
\newpage


%----------------------------------------------------------------------------------------
% ÍNDICE
%----------------------------------------------------------------------------------------	
\tableofcontents
\clearpage


%----------------------------------------------------------------------------------------
% 1. IDENTIFICACIÓN DEL PROYECTO
%----------------------------------------------------------------------------------------
\renewcommand{\thepage}{\arabic{page}}
\section{Identificación del proyecto}
\input{tex/01_identificacion_del_proyecto}


%----------------------------------------------------------------------------------------
% 2. DATOS DEL LICITADOR
% 2.1 DENOMINACIÓN DE LA EMPRESA
% 2.2 HISTORIAL DEL EQUIPO
%----------------------------------------------------------------------------------------
\section{Datos del licitador}
\input{tex/02_0_datos_del_licitador}

\subsection{Denominación de la empresa}
\input{tex/02_1_denominacion_de_la_empresa}

\subsection{Historial del equipo}
\begin{itemize}

\item El equipo cuenta con experiencia en el mantenimiento de administración de sistemas.
\item El equipo cuenta con experiencia en el desarrollo y mantenimiento de sistemas web utilizando diferentes lenguajes como HTML, PHP, CSS, JavaScript.
\item El equipo cuenta con experiencia en la administración de gestores de contenidos como Drupal, Wordpress... 
\item El equipo cuenta con experiencia en el desarrollo y mantenimiento de bases de datos.
\item El equipo a montado web con funcionalidades similares con gestión de contenido como películas o series.

\end{itemize}


%----------------------------------------------------------------------------------------
% 3. OBJETO DE LA PROPUESTA
%----------------------------------------------------------------------------------------
\section{Objeto de la propuesta}
\input{tex/03_objeto_de_la_propuesta}


%----------------------------------------------------------------------------------------
% 4. REQUISITOS DEL SISTEMA
%----------------------------------------------------------------------------------------
\section{Requisitos del sistema}
\documentclass[10pt,a4paper]{article}
\usepackage[T1]{fontenc}
\usepackage[utf8]{inputenc}
\usepackage[spanish]{babel}
\begin{document}

\section{Requisitos}
\begin{enumerate}

\item Un microcontrolador (elemento) estará compuesto de los siguientes campos: 
	\begin{enumerate}
		\item Referencia (será única para cada elemento).
        \item Arquitectura
        \item Frecuencia
        \item Flash
        \item RAM
        \item Precio
	\end{enumerate}
	
\item Insertar un elemento en el carro de compra.
   
\item Eliminar un elemento del carro de compra.
	
\item Modificar un elemento del carro de compra. Por modificar se entiende alterar el número de unidades de los elementos.

\item Se podrá acceder a los elementos del catálogo mediante un listado en el que aparezcan todos sus elementos.

\item Se podrá actualizar varios elementos del carro de manera simultánea. Por actualizar se entiende a recalcular los precios de cada artículo en el caso de que éstos hayan sido modificados.

\item Se permitirá realizar búsquedas de productos en función de un único campo de búsqueda y en base a una de las características de los elementos. 
	
\item Los resultados de la búsqueda se presentarán como un listado (sin paginación) que mostrará, de cada elemento, todos sus campos en columnas.

\item Se permitirá realizar pedidos en los que se incluirán los datos del cliente cada vez. Es decir, no existirá persistencia de los datos del cliente tras realizar pedidos.

\item Los pedidos contendrán la suficiente información para identificar a los clientes. Además, no permitirán la reserva de los productos solicitados, únicamente generarán un presupuesto con el coste de los productos elegidos.

\item Los datos solicitados del cliente para los pedidos serán los siguientes:

	\begin{enumerate}
		\item Nombre 
		\item Apellidos
	    \item Dirección
	    \item Ciudad
	    \item Provincia
	    \item País
	    \item Código postal
	    \item Teléfono
	    \item Correo electrónico
	    \item CIF y Empresa aparecerán como campos opcionales que servirán de distinción entre particulares y entidades.
	\end{enumerate}
	

\item Se contará con una vista diferente para la administración del catálogo a la que no podrán acceder los clientes (se ejecutará solamente en local). 
En ésta vista se podrán realizar las acciones de:
	
	\begin{enumerate}
    	\item Insertar un elemento en el catálogo (a partir de las
           arquitecturas disponibles, no se permitirá añadir nuevas
           arquitecturas al catálogo).  
		\item Eliminar un elemento del catálogo.
        \item Modificar un elemento del catálogo. En éste caso, modificar un elemento del catálogo sería cambiar cualquiera de sus características.
    \end{enumerate}   

\end{enumerate}
\end{document}



%----------------------------------------------------------------------------------------
% 5. DESCRIPCIÓN TÉCNICA DE LA SOLUCIÓN PROPUESTA
% 5.1. ESQUEMA DE LA ARQUITECTURA
% 5.2. TECNOLOGÍAS 
%----------------------------------------------------------------------------------------
\section{Descripción técnica de la solución}
\input{tex/05_0_descripcion_tecnica_de_la_solucion}

\subsection{Denominación de la empresa}
\input{tex/05_1_denominacion_de_la_empresa}

\subsection{Historial del equipo}
\input{tex/05_2_historial_del_equipo}


%----------------------------------------------------------------------------------------
% 6. METODOLOGÍA DE GESTIÓN Y PLANIFICACIÓN
% 6.1. FASES Y ACTIVIDADES DEL PROYECTO
% 6.2. RECURSOS HUMANOS
% 6.3. CRONOGRAMA
% 6.4. PRESUPUESTO
%----------------------------------------------------------------------------------------
\subsection{Presupuesto}
\input{tex/06_0_metodologia_de_gestion_y_planificacion}

\subsection{Denominación de la empresa}
\input{tex/06_1_fases_y_actividades_del_proyecto}

\subsection{Historial del equipo}
\includepdf[1-1]{cv_pdf/alberto_berbel_curriculum_vitae.pdf}
\includepdf[1-1]{cv_pdf/alejandro_gracia_mateo_curriculum_vitae.pdf}
\includepdf[1-1]{cv_pdf/javier_curriculum_vitae.pdf}
\includepdf[1-1]{cv_pdf/daniel_curriculum_vitae.pdf}
\includepdf[1-1]{cv_pdf/hector_curriculum_vitae.pdf}
\includepdf[1-1]{cv_pdf/simon_curriculum_vitae.pdf}

\subsection{Denominación de la empresa}
\begin{figure}
\centering
\scalebox{0.75}{
  \begin{gantt}[xunitlength=0.75cm,fontsize=\small,titlefontsize=\small]{12}{25}
	  
    \begin{ganttitle} % Primer título año: 2014
      \titleelement{2014}{25}
    \end{ganttitle}
	
    \begin{ganttitle} % Segundo título meses: Ene-Junio
	  \titleelement{Febrero}{5}
	  \titleelement{Marzo}{5}
	  \titleelement{Abril}{5}
	  \titleelement{Mayo}{5}
	  \titleelement{Junio}{5}
    \end{ganttitle}
	
    \begin{ganttitle} % Tercer título fechas importantes
	  
	  % Febrero
	  \titleelement{ }{4}
	  \titleelement{28}{1}
	  
	  % Marzo
	  \titleelement{ }{1}
	  \titleelement{10}{1}
	  \titleelement{ }{1}
	  \titleelement{20}{1}
	  \titleelement{27}{1}
	  
	  % Abril
	  \titleelement{ }{1}
	  \titleelement{10}{1}
	  \titleelement{ }{1}
	  \titleelement{24}{1}
	  \titleelement{ }{2}
	  
	  % Mayo
	  \titleelement{1}{1}
	  \titleelement{2}{1}
	  \titleelement{3}{1}
	  \titleelement{4}{1}
	  \titleelement{5}{1}
	  \titleelement{5}{1}
	  
	  % Junio
	  \titleelement{1}{1}
	  \titleelement{2}{1}
	  \titleelement{3}{1}
	  \titleelement{4}{1}
	  \titleelement{5}{1}
	  \titleelement{5}{1}

    \end{ganttitle}
	
	\ganttgroup{1\textsuperscript{a} iteración}{4}{18}
    \ganttbar[pattern=crosshatch,color=verde_P1]{Tarea 0}{4}{4}
	\ganttbar[pattern=crosshatch,color=rojo_P1]{Tarea 1}{6}{4}
	\ganttbar[pattern=crosshatch,color=rojo_P1]{Tarea 2}{6}{4}
	
    \ganttbar{task 2}{5}{10}
    \ganttbar[pattern=vertical lines,color=blue]{task 3}{15}{3}
    \ganttbar{task 4}{20}{3}
    \ganttcon{15}{4}{20}{6}
    \ganttbar{task 5}{15}{5}
  \end{gantt}
}
\caption{Diagrama de Gantt}
\label{fig:diagGantt}
\end{figure}

\subsection{Historial del equipo}
\paragraph{}Se presenta a continuación la propuesta de presupuesto u oferta económica:
\includepdf[pages=-]{oferta_economica.pdf}

\end{document}
