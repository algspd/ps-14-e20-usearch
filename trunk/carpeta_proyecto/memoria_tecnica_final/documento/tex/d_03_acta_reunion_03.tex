% D.03. ACTA DE REUNION #03
%----------------------------------------------------------------------------------------------
\begin{center}	
\Large{Acta de Reunión Nº03\hspace{0.25em}-\hspace{0.25em}\tituloReunion}
\end{center}
\vspace{1.5em}

% PRIMERA TABLA: INFORMACIÓN BÁSICA
\begin{longtable}{ | L{\dimexpr 0.420\linewidth-2\tabcolsep} |
				     L{\dimexpr 0.570\linewidth-2\tabcolsep} | }
\hline % ------------------------------------------------------------------------
\rowcolor{grisCabeceraTabla}
\mc{2}{\bf Información básica}  \\
%\hline % ------------------------------------------------------------------------
%{\bf Cliente} & Nombre del cliente (por definir o no es necesario ?)  \\
\hline % ------------------------------------------------------------------------
{\bf Proyecto} & $\mu$Search \\ 
\hline % ------------------------------------------------------------------------
{\bf Fecha y hora de comienzo} & 26/02/14 - 20:00 \\
\hline % ------------------------------------------------------------------------
{\bf Lugar} & Seminario 21, Edif. Ada Byron. EINA. Unizar \\
\hline % ------------------------------------------------------------------------
{\bf Tipo de reunión} & Estándar \\
\hline % ------------------------------------------------------------------------
\end{longtable}


%----------------------------------------------------------------------------------------
% SEGUNDA TABLA - ASISTENTES
\begin{longtable}{ | C{\dimexpr 0.070\linewidth-2\tabcolsep} |
                     L{\dimexpr 0.350\linewidth-2\tabcolsep} |
                     C{\dimexpr 0.370\linewidth-2\tabcolsep} |
                     C{\dimexpr 0.200\linewidth-2\tabcolsep} | }
\hline % ------------------------------------------------------------------------
\rowcolor{grisCabeceraTabla}
\mc{4}{\bf Asistentes} \\
\hline % ------------------------------------------------------------------------
{\bf Nº} & {\bf Nombre y Apellidos} & {\bf Cargo} & {\bf Rol} \\
\hline % ------------------------------------------------------------------------
{\bf 1} & Alberto Berbel Aznar & Verificación y validación & --  \\
\hline % ------------------------------------------------------------------------
{\bf 2} & Javier Briz Alastrué & Gestor de configuraciones & Cronometrador  \\
\hline % ------------------------------------------------------------------------
{\bf 3} & Héctor Francia Molinero & Gestor de calidad & Observador  \\
\hline % ----------------------------------------------------
{\bf 4} & Daniel García Páez & Director del proyecto & Director \\
\hline % ------------------------------------------------------------------------
{\bf 5} & Alejandro Gracia Mateo & Gestor de planificación & Planificador  \\
\hline % ------------------------------------------------------------------------
{\bf 6} & Simón Ortego Parra & Gestor de desarrollo & Secretario  \\
\hline % ------------------------------------------------------------------------
\end{longtable}

%----------------------------------------------------------------------------------------
% CUARTA TABLA - OBJETIVOS, CUERPO DE LA REUNIÓN, DECISIONES TOMADAS, ...
\begin{longtable}{ | C{\dimexpr\linewidth-2\tabcolsep} | }
\hline % ------------------------------------------------------------------------
\rowcolor{grisCabeceraTabla}
{\bf Objetivos} \\
\hline % ------------------------------------------------------------------------
\itemNvlUno{Fijar el nombre del proyecto y el logotipo.}
\itemNvlUno{Ajustar la agenda semanal y los roles de cada miembro para 
las reuniones de manera definitiva.}
\itemNvlUno{Establecer los requisitos de la aplicación.}
\itemNvlUno{Repartición de tareas para la propuesta de proyecto.} \\
\endfirsthead
\hline % ------------------------------------------------------------------------
\endhead
\espacioSubtabla
\hline % ------------------------------------------------------------------------
\endfoot
\hline % ------------------------------------------------------------------------
\endlastfoot
\hline % ------------------------------------------------------------------------
\rowcolor{grisCabeceraTabla}
{\bf Cuerpo de la reunión} \\
\hline % ------------------------------------------------------------------------
\itemNvlUno{Se asignaron los roles de las reuniones de la siguiente manera. Los
roles de secretario, director y planificador serán fijos para todas las reuniones,
el resto de roles se rotarán.}
\itemNvlUno{Por otro lado, se decidió que el nombre del proyecto y el logotipo
se decidirían a través de una encuesta online.}
\itemNvlUno{A continuación, se examinó el avance individual de cada uno de 
los miembros del equipo. También, se revisó lo que faltaba por realizar de
las dos últimas reuniones.}
\itemNvlUno{Luego, se plantearon a muy alto nivel los requisitos para la 
aplicación.}
\itemNvlUno{Además, se anotó un requisito importante de la aplicación.}\\
\rowcolor{grisCabeceraTabla}
{\bf Decisiones tomadas}\\
\hline % ------------------------------------------------------------------------
\itemNvlUno{Se plantearon los siguientes requisitos de aplicación:}
    \itemNvlDos{Insertar un elemento en el catálogo.}
    \itemNvlDos{Eliminar un elemento del catálogo.}
    \itemNvlDos{Modificar un elemento del catálogo.}
    \itemNvlDos{Insertar un elemento en el carrito de compra.}
    \itemNvlDos{Eliminar un elemento del carrito de compra.}
    \itemNvlDos{Modificar un elemento del carrito de compra.}
    \itemNvlDos{Actualizar varios elementos del carro de manera simultánea.}
    \itemNvlDos{Búsqueda de productos atendiendo a un único campo de búsqueda
    en cada caso.}
\itemNvlUno{La restricción para la aplicación que se propuso es el siguiente:
``Se permitirá realizar pedidos que incluirán los datos del cliente cada vez. 
Es decir, no existirá persistencia de los datos del cliente tras la realización 
de los pedidos. Los pedidos contendrán la suficiente información para identificar 
a los clientes. Además, no permitirán la reserva de los productos solicitados, 
únicamente se generará un presupuesto del coste de los productos elegidos''.}\\
\hline % ------------------------------------------------------------------------
\rowcolor{grisCabeceraTabla}
{\bf Próxima reunión prevista} \\
\hline % ------------------------------------------------------------------------
\itemNvlUno{Se decidirá tras realizarse la cuarta reunión, que tiene como fin el seguimiento
del lanzamiento de la primera iteración del proyecto.}\\
\end{longtable}

