% D.02. ACTA DE REUNION #02
%----------------------------------------------------------------------------------------------
\begin{center}	
\Large{Acta de Reunión Nº02\hspace{0.25em}-\hspace{0.25em}\tituloReunion}
\end{center}
\vspace{1.5em}

% PRIMERA TABLA: INFORMACIÓN BÁSICA
\begin{longtable}{ | L{\dimexpr 0.420\linewidth-2\tabcolsep} |
				     L{\dimexpr 0.570\linewidth-2\tabcolsep} | }
\hline % ------------------------------------------------------------------------
\rowcolor{grisCabeceraTabla}
\mc{2}{\bf Información básica}  \\
%\hline % ------------------------------------------------------------------------
%{\bf Cliente} & Nombre del cliente (por definir o no es necesario ?)  \\
\hline % ------------------------------------------------------------------------
{\bf Proyecto} & $\mu$Search \\ 
\hline % ------------------------------------------------------------------------
{\bf Fecha y hora de comienzo} & 20/02/14 - 12:00 \\
\hline % ------------------------------------------------------------------------
{\bf Lugar} & Seminario 21, Edif. Ada Byron. EINA. Unizar \\
\hline % ------------------------------------------------------------------------
{\bf Tipo de reunión} & Estándar \\
\hline % ------------------------------------------------------------------------
\end{longtable}


%----------------------------------------------------------------------------------------
% SEGUNDA TABLA - ASISTENTES
\begin{longtable}{ | C{\dimexpr 0.070\linewidth-2\tabcolsep} |
                     L{\dimexpr 0.350\linewidth-2\tabcolsep} |
                     C{\dimexpr 0.370\linewidth-2\tabcolsep} |
                     C{\dimexpr 0.200\linewidth-2\tabcolsep} | }
\hline % ------------------------------------------------------------------------
\rowcolor{grisCabeceraTabla}
\mc{4}{\bf Asistentes} \\
\hline % ------------------------------------------------------------------------
{\bf Nº} & {\bf Nombre y Apellidos} & {\bf Cargo} & {\bf Rol} \\
\hline % ------------------------------------------------------------------------
{\bf 1} & Alberto Berbel Aznar & Verificación y validación & --  \\
\hline % ------------------------------------------------------------------------
{\bf 2} & Javier Briz Alastrué & Gestor de configuraciones & --  \\
\hline % ------------------------------------------------------------------------
{\bf 3} & Héctor Francia Molinero & Gestor de calidad & --  \\
\hline % ----------------------------------------------------
{\bf 4} & Daniel García Páez & Director del proyecto & -- \\
\hline % ------------------------------------------------------------------------
{\bf 5} & Alejandro Gracia Mateo & Gestor de planificación & --  \\
\hline % ------------------------------------------------------------------------
{\bf 6} & Simón Ortego Parra & Gestor de desarrollo & Secretario  \\
\hline % ------------------------------------------------------------------------
\end{longtable}

%----------------------------------------------------------------------------------------
% CUARTA TABLA - OBJETIVOS, CUERPO DE LA REUNIÓN, DECISIONES TOMADAS, ...
\begin{longtable}{ | C{\dimexpr\linewidth-2\tabcolsep} | }
\hline % ------------------------------------------------------------------------
\cabeceraTabla{Objetivos} \\
\hline % ------------------------------------------------------------------------
\endfirsthead
\hline % ------------------------------------------------------------------------
\endhead
\espacioSubtabla
\hline % ------------------------------------------------------------------------
\endfoot
\hline % ------------------------------------------------------------------------
\endlastfoot

\itemNvlUno{Organizar de forma consensuada el comienzo del proyecto y algunos
aspectos generales.} \\

\hline % ------------------------------------------------------------------------
\cabeceraTabla{Cuerpo de la reunión} \\
\hline % ------------------------------------------------------------------------

\itemNvlUno{Cada uno de los asistentes se presentó a la hora.}
\itemNvlUno{Se analizaron los requisitos de la aplicación de nuevo y
se fijaron perfectamente todos los parámetros.}
\itemNvlUno{Se pasó a hablar de la arquitectura en alto nivel del sistema
a desarrollar y de realizar los diagramas explicativos apropiados (diagramas
de componentes y de despliegue).}
\itemNvlUno{Se establecieron las tecnologías y estándares de codificación del
proyecto software.}
\itemNvlUno{Se decidieron las herramientas usadas para la documentación durante 
el desarrrollo del proyecto.}
\itemNvlUno{Se establecieron los mecanismos de comunicación entre los componentes
del equipo.}
\itemNvlUno{Se realizó una planificación de las tareas ha realizar inmediatamente
para el desarrollo del documento de ``Propuesta de Proyecto''.}\\

\hline % ------------------------------------------------------------------------
\cabeceraTabla{Decisiones tomadas}  \\
\hline % ------------------------------------------------------------------------

\itemNvlUno{Se establecieron como tecnologías y estándares de codificación (los
establecidos por los desarrolladores de éstas) las siguientes:}
	\itemNvlDos{HTML 5.}
	\itemNvlDos{CSS 3.}
	\itemNvlDos{PHP 5 y CodeIgniter (la versión apropiada).}
	\itemNvlDos{MySQL.}
	\itemNvlDos{LaTex: para la generación de la órden de pedido automática.}
\itemNvlUno{Para la documentación durante el desarrrollo del proyecto
se van a utilizar las siguientes herramientas:}
	\itemNvlDos{LaTex: para todo tipo de documentación (Actas, Convocatorias...)}
	\itemNvlDos{Excel o similar: para la anotación de los esfuerzos
	de los miembros del equipo.}
	\itemNvlDos{Modelio: para realizar diferentes diagramas de la arquitectura del software.}
	\itemNvlDos{Subversion: para el almacenamiento de todos los documentos.}
\itemNvlUno{Para la comunicación entre los miembros se van a utilizar:}
	\itemNvlDos{Google Groups: para todo tipo de comunicación.}
	\itemNvlDos{Whatsapp: para la comunicación informal (problemas, retrasos, citas,...)}
\itemNvlUno{Se realizó la siguiente planificación de cara a la presentación de la
``Propuesta de Proyecto'' que incluye una aproximación del número de horas y
los miembros a los que se les asigna cada una de las tareas:}
	\itemNvlDos{[2 h.] - Diagrama E/R de la BD + prototipo de la 
	aplicación en papel: Daniel y Javier.}
	\itemNvlDos{[1 h.] - Diseño arquitectural del sistema (diag. componentes 
	y despliegue): Alberto.}
	\itemNvlDos{[1 h.] - Plantillas de documentación en LaTeX para las actas,
	convocatorias y propuesta de proyecto: Simón.}
	\itemNvlDos{[1 h.] - Hojas de esfuerzos: Códigos de categorías (incluyendo un 
	documento de texto explicativo con código + nombre + descr.): Alejandro}
	\itemNvlDos{[2 h.] - Logotipo + Nombre de la empresa: Héctor.}\\

\hline % ------------------------------------------------------------------------
\cabeceraTabla{Temas pendientes} \\
\hline % ------------------------------------------------------------------------

\itemNvlUno{Elección del responsable de la propuesta de proyecto.}
\itemNvlUno{Calendario de trabajo.}
\itemNvlUno{Iniciar el uso de la Wiki + Subversion.} \\
	
\hline % ------------------------------------------------------------------------
\cabeceraTabla{Próxima reunión prevista} \\
\hline % ------------------------------------------------------------------------
Miércoles 26 de Febrero del 2014 a las 20:00 \\
\end{longtable}
