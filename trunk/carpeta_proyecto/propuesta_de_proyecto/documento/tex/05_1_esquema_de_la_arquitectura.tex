\paragraph{}El patrón que vamos a utilizar para el diseño arquitectural de nuestro 
catálogo va a ser el de Modelo-Vista-Controlador, en contreto, la variante Modelo-Vista-Presentador.

\vspace{.2cm}
\begin{figure}[h!]
 \centering
 \includegraphics[scale=.6]{img/diagrama_desplieque}
 \caption{Diagrama de despliegue del sistema}
\end{figure}
\vspace{.2cm}


\paragraph{Interfaz web}Es nuestro componente vista, lo que utilizan los usuarios para 
interactuar con la aplicación y visualizar los resultados que producen dichas
interacciones. Las acciones del usuario que impliquen el acceso o la modificación
de los datos del modelo son delegadas al componente Controlador.

\paragraph{Controlador}Es nuestro componente Presentador, tiene toda la lógica de la vista 
y es responsable de sincronizar el modelo y la vista. Cuando la vista notifica el 
presentador que el usuario ha hecho algo (por ejemplo, hacer clic en un botón), 
el presentador a continuación, actualiza el modelo y sincroniza los cambios 
entre el modelo y la vista.

\paragraph{Es Base de datos}Es nuestro componente modelo, se encarga de encapsular 
los datos y ofrecer operaciones para su acceso y procesamiento. Solo el componente
Controlador interactúa con este componente.