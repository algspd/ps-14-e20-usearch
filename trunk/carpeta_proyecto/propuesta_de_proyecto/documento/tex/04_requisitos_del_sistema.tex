\documentclass[10pt,a4paper]{article}
\usepackage[T1]{fontenc}
\usepackage[utf8]{inputenc}
\usepackage[spanish]{babel}
\begin{document}

\section{Requisitos}
\begin{enumerate}

\item Un microcontrolador (elemento) estará compuesto de los siguientes campos: 
	\begin{enumerate}
		\item Referencia (será única para cada elemento).
        \item Arquitectura
        \item Frecuencia
        \item Flash
        \item RAM
        \item Precio
	\end{enumerate}
	
\item Insertar un elemento en el carro de compra.
   
\item Eliminar un elemento del carro de compra.
	
\item Modificar un elemento del carro de compra. Por modificar se entiende alterar el número de unidades de los elementos.

\item Se podrá acceder a los elementos del catálogo mediante un listado en el que aparezcan todos sus elementos.

\item Se podrá actualizar varios elementos del carro de manera simultánea. Por actualizar se entiende a recalcular los precios de cada artículo en el caso de que éstos hayan sido modificados.

\item Se permitirá realizar búsquedas de productos en función de un único campo de búsqueda y en base a una de las características de los elementos. 
	
\item Los resultados de la búsqueda se presentarán como un listado (sin paginación) que mostrará, de cada elemento, todos sus campos en columnas.

\item Se permitirá realizar pedidos en los que se incluirán los datos del cliente cada vez. Es decir, no existirá persistencia de los datos del cliente tras realizar pedidos.

\item Los pedidos contendrán la suficiente información para identificar a los clientes. Además, no permitirán la reserva de los productos solicitados, únicamente generarán un presupuesto con el coste de los productos elegidos.

\item Los datos solicitados del cliente para los pedidos serán los siguientes:

	\begin{enumerate}
		\item Nombre 
		\item Apellidos
	    \item Dirección
	    \item Ciudad
	    \item Provincia
	    \item País
	    \item Código postal
	    \item Teléfono
	    \item Correo electrónico
	    \item CIF y Empresa aparecerán como campos opcionales que servirán de distinción entre particulares y entidades.
	\end{enumerate}
	

\item Se contará con una vista diferente para la administración del catálogo a la que no podrán acceder los clientes (se ejecutará solamente en local). 
En ésta vista se podrán realizar las acciones de:
	
	\begin{enumerate}
    	\item Insertar un elemento en el catálogo (a partir de las
           arquitecturas disponibles, no se permitirá añadir nuevas
           arquitecturas al catálogo).  
		\item Eliminar un elemento del catálogo.
        \item Modificar un elemento del catálogo. En éste caso, modificar un elemento del catálogo sería cambiar cualquiera de sus características.
    \end{enumerate}   

\end{enumerate}
\end{document}
