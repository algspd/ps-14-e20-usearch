\paragraph{} Se presenta en este documento la propuesta de proyecto  para el desarrollo de una aplicación web para la venta de microcontroladores a través de un catálogo electrónico, tal y como nos pedía el cliente.
\paragraph{} La aplicación web permite al cliente la búsqueda de microcontroladores en un catálogo electrónico existente, mostrando como resultado un listado sin paginación. 
\paragraph{} El cliente puede añadir desde dicho listado a su carrito de compra los microcontroladores que desee y modificar posteriormente las unidades finales en el carrito de compra, para así poder pedir finalmente la generación de un presupuesto en formato PDF. Cabe recalcar que cada vez que un cliente desee generar un pedido debe introducir sus datos personales y de empresa, es decir, no hay persistencia de los datos.
\paragraph{} Se proporciona además una interfaz web exclusiva para los administradores, a la que solo se accederá desde la empresa cliente de manera local, que tendrán permiso para añadir nuevos microcontroladores al catálogo y modificar y/o eliminar los ya existentes.
\paragraph{} La aplicación entregada a la empresa cliente incluye el servidor web y de base de datos en completo funcionamiento, siendo esto una gran aliciente para la adquisición del producto, pues evita un gasto importante a la empresa cliente. 
\paragraph{} El diseño arquitectural de la aplicación está basado en el patrón Modelo-Vista-Controlador, en concreto, la variante Modelo-Vista-Presentador, separando así de manera notable la vista de la aplicación de los datos de la aplicación, escondiendo así al usuario todos los detalles internos para mayor seguridad de la aplicación.
\paragraph{} En cuanto a detalles tecnológicos de la implementación, concretar que la interfaz web estará implementada con HTML5, CSS3 integrado con el control implementado en PHP5 utilizando el framework CodeIgniter2.1.4 y que se comunica con una base de datos en MySQL5.5. Se asegura además que la aplicación funcionará correctamente en varios de los navegadores web más actuales (Chrome, Mozilla, Opera…).
\paragraph{} Ya presentados los detalles de funcionamiento e implementación de la aplicación, se presentan en el punto final del documento:
\begin{itemize}
\item La división de tareas del proyecto junto a un datagrama, proveyendo así al cliente de una idea bastante exacta de los cumplimientos de plazos previstos.
\item Una oferta económica del desarrollo total de la aplicación.
\end{itemize}