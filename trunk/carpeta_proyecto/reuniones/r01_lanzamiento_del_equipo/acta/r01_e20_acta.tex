\documentclass[10pt,spanish]{article}

%----------------------------------------------------------------------------------------
% Codificación, y usar una fuente similar a la Palatino (y no Latin Modern Roman)
\usepackage[utf8]{inputenc} % Acentos, etc.
\usepackage[T1]{fontenc}
\usepackage[spanish]{babel} % Castellano
\usepackage{tgpagella}      % Fuente similar a Palatino

%----------------------------------------------------------------------------------------
% MÁRGENES (menores que en otro tipo de documentos)
\addtolength{\oddsidemargin}{-.375in}
\addtolength{\evensidemargin}{-.375in}
\addtolength{\textwidth}{1.25in}

%\addtolength{\topmargin}{-.375in}
%\addtolength{\textheight}{.5in}

%----------------------------------------------------------------------------------------
% COLORES
\usepackage{xcolor,colortbl} % Colores personalizados y color fondo tablas
\definecolor{grisCabeceraTabla}{RGB}{220,220,220}
\definecolor{grisHeader}{RGB}{180,180,180}
	
%----------------------------------------------------------------------------------------
% TABLAS	  
%\usepackage{supertabular} % Tablas multipágina
\linespread {1.4} % Tamaño que ocupa una línea (celda)

\usepackage{array}
\usepackage{longtable}
\setlength\LTleft{0pt}
\setlength\LTright{0pt}
\setlength{\columnsep}{0em}

% Definir los estilos de las celdas para indentar el texto
\newcolumntype{L}[1]{>{\raggedright\let\newline\\\arraybackslash\hspace{0pt}}m{#1}}
\newcolumntype{C}[1]{>{\centering\let\newline\\\arraybackslash\hspace{0pt}}m{#1}}
\newcolumntype{R}[1]{>{\raggedleft\let\newline\\\arraybackslash\hspace{0pt}}m{#1}}

% Filas con columnas multiples
\newcommand{\mc}[2]{\multicolumn{#1}{|C{\dimexpr 1\linewidth-2\tabcolsep}|}{#2}}

% Colores de lineas de las tablas
%\arrayrulecolor{grisCabeceraTabla}
\arrayrulecolor{grisHeader}

% Eliminar espacio previo en lista de ítems
\usepackage{enumitem}
\setlist{nolistsep}

% Comandos especiales para crear items dentro de tabla y poder hacer
% una tabla de multiples paginas y que se pueda realizar el salto de
% pagina en cualquier punto.
\usepackage{scrextend}
\newenvironment{lvlOneItem}
	{\begin{addmargin}[2.5em]{1em}
	\vspace{-1em}
	\hspace{-1em}$\bullet$\hspace{0.5em}}
	{\vspace{-2em}\end{addmargin}}
\newenvironment{lvlTwoItem}
	{\begin{addmargin}[4em]{1em}
	\vspace{-1em}
	\hspace{-1em}$\circ$\hspace{0.5em}}
	{\vspace{-2em}\end{addmargin}}

\newcommand{\itemNvlUno}[1]{\begin{lvlOneItem}{#1}\end{lvlOneItem}\\}
\newcommand{\itemNvlDos}[1]{\begin{lvlTwoItem}{#1}\end{lvlTwoItem}\\}
\newcommand{\espacioSubtabla}{\\[0.5ex]}
\newcommand{\cabeceraTabla}[1]{\rowcolor{grisCabeceraTabla}{\bf #1}}

%----------------------------------------------------------------------------------------
% METADATOS DEL PDF Y PDF CLICKEABLE
\usepackage{hyperref}
\usepackage{hyperxmp}

% DATOS A CAMBIAR
\newcommand{\numeroDeReunion}{01}
\newcommand{\tituloReunion}{\bf Lanzamiento del equipo}
\newcommand{\nombreDelProyecto}{$\mu$Search}

\hypersetup{
	pdfauthor={Alberto Berbel Aznar, 
				Javier Briz Alastrué, 
				Héctor Francia Molinero, 
				Daniel García Páez,
				Alejandro Gracia Mateo,
				Simón Ortego Parra},
	pdftitle={E20 - Acta R\numeroDeReunion: \tituloReunion},
	pdfsubject={Proyecto Software. Grado Ing. Informática. EINA. Unizar},
	pdfkeywords={},
	pdfcopyright={Copyright (C) 2014 by Alberto Berbel Aznar, 
				Javier Briz Alastrué, 
				Héctor Francia Molinero, 
				Daniel García Páez,
				Alejandro Gracia Mateo,
				Simón Ortego Parra. All rights reserved.},
	pdfproducer={PDFLatex},
	pdfcreator={ps2pdf},
	colorlinks=false
}

%----------------------------------------------------------------------------------------
% Encabezados y pies de pagina : FancyHdr
\usepackage{fancyhdr}
\pagestyle{fancy}
\fancyhf{} % borrar todos los ajustes
\setlength{\headheight}{15pt}
\usepackage{lastpage} % Para poner total de paginas en el footer, ej: pag 1/4

\fancyhead[L]{\color{grisHeader}{\large BITPARTY}}
\fancyhead[R]{\color{grisHeader}E20 - Acta R\numeroDeReunion \\ \nombreDelProyecto}
\fancyfoot[R]{\color{grisHeader}\thepage/\pageref*{LastPage}}

% Modifica el ancho de las líneas de cabecera y pie
\renewcommand{\headrulewidth}{0pt}
\renewcommand{\footrulewidth}{0pt}
\renewcommand{\headsep}{0.6in}
\renewcommand{\headwidth}{6in}

%----------------------------------------------------------------------------------------
%----------------------------------------------------------------------------------------
% INICIO DEL DOCUMENTO
%----------------------------------------------------------------------------------------

\begin{document}
	
\begin{center}	
\Large{Acta de Reunión Nº \numeroDeReunion\hspace{0.25em}-\hspace{0.25em}\tituloReunion}
\end{center}
\vspace{1.5em}

% PRIMERA TABLA: INFORMACIÓN BÁSICA
\begin{longtable}{ | L{\dimexpr 0.420\linewidth-2\tabcolsep} |
				     L{\dimexpr 0.570\linewidth-2\tabcolsep} | }
\hline % ------------------------------------------------------------------------
\rowcolor{grisCabeceraTabla}
\mc{2}{\bf Información básica}  \\
%\hline % ------------------------------------------------------------------------
%{\bf Cliente} & Nombre del cliente (por definir o no es necesario ?)  \\
\hline % ------------------------------------------------------------------------
{\bf Proyecto} & $\mu$Search \\ 
\hline % ------------------------------------------------------------------------
{\bf Fecha y hora de comienzo} & 20/02/14 - 12:00 \\
\hline % ------------------------------------------------------------------------
{\bf Lugar} & Seminario 21, Edif. Ada Byron. EINA. Unizar \\
\hline % ------------------------------------------------------------------------
{\bf Tipo de reunión} & Estándar \\
\hline % ------------------------------------------------------------------------
\end{longtable}


%----------------------------------------------------------------------------------------
% SEGUNDA TABLA - ASISTENTES
\begin{longtable}{ | C{\dimexpr 0.070\linewidth-2\tabcolsep} |
                     L{\dimexpr 0.350\linewidth-2\tabcolsep} |
                     C{\dimexpr 0.370\linewidth-2\tabcolsep} |
                     C{\dimexpr 0.200\linewidth-2\tabcolsep} | }
\hline % ------------------------------------------------------------------------
\rowcolor{grisCabeceraTabla}
\mc{4}{\bf Asistentes} \\
\hline % ------------------------------------------------------------------------
{\bf Nº} & {\bf Nombre y Apellidos} & {\bf Cargo} & {\bf Rol} \\
\hline % ------------------------------------------------------------------------
{\bf 1} & Alberto Berbel Aznar & Verificación y validación & --  \\
\hline % ------------------------------------------------------------------------
{\bf 2} & Javier Briz Alastrué & Gestor de configuraciones & --  \\
\hline % ------------------------------------------------------------------------
{\bf 3} & Héctor Francia Molinero & Gestor de calidad & --  \\
\hline % ----------------------------------------------------
{\bf 4} & Daniel García Páez & Director del proyecto & -- \\
\hline % ------------------------------------------------------------------------
{\bf 5} & Alejandro Gracia Mateo & Gestor de planificación & --  \\
\hline % ------------------------------------------------------------------------
{\bf 6} & Simón Ortego Parra & Gestor de desarrollo & Secretario  \\
\hline % ------------------------------------------------------------------------
{\bf 7} & Fco. Javier Zarazaga Soria & -- & Preparador  \\
\hline % ------------------------------------------------------------------------
\end{longtable}


%----------------------------------------------------------------------------------------
% TERCERA TABLA - AUSENTES
%\begin{longtable}{ | C{\dimexpr 0.070\linewidth-2\tabcolsep} |
%                     L{\dimexpr 0.350\linewidth-2\tabcolsep}  |
%                     C{\dimexpr 0.570\linewidth-2\tabcolsep} | }
%\hline % ------------------------------------------------------------------------
%\rowcolor{grisCabeceraTabla}
%\mc{3}{\bf Ausentes} \\ 
%\hline % ------------------------------------------------------------------------
%{\bf Nº} & {\bf Nombre y Apellidos} & {\bf Cargo} \\
%\hline % ------------------------------------------------------------------------
%{\bf 1} & Alberto Berbel Aznar & Verificación y validación \\
%\hline % ------------------------------------------------------------------------
%{\bf 1} & Javier Briz Alastrué & Gestor de configuraciones \\
%\hline % ----------------------------------------------------
%{\bf 2} & Héctor Francia Molinero & Gestor de calidad \\
%\hline % ------------------------------------------------------------------------
%{\bf 2} & Daniel García Páez & Director del proyecto \\
%\hline % ------------------------------------------------------------------------
%{\bf 3} & Alejandro Gracia Mateo & Gestor de planificación \\
%\hline % ------------------------------------------------------------------------
%{\bf 4} & Simón Ortego Parra & Gestor de desarrollo \\
%\hline % ------------------------------------------------------------------------
%\end{longtable}


%----------------------------------------------------------------------------------------
% CUARTA TABLA - OBJETIVOS, CUERPO DE LA REUNIÓN, DECISIONES TOMADAS, ...
\begin{longtable}{ | C{\dimexpr\linewidth-2\tabcolsep} | }
\hline % ------------------------------------------------------------------------
\cabeceraTabla{Objetivos} \\
\hline % ------------------------------------------------------------------------
\endfirsthead
\hline % ------------------------------------------------------------------------
\endhead
\espacioSubtabla
\hline % ------------------------------------------------------------------------
\endfoot
\hline % ------------------------------------------------------------------------
\endlastfoot

\itemNvlUno{Establecer las políticas y aspectos de organización generales comunes
para un buen funcionamiento de equipo.}\\

\hline % ------------------------------------------------------------------------
\cabeceraTabla{Cuerpo de la reunión} \\
\hline % ------------------------------------------------------------------------
\itemNvlUno{Previamente a realizar la reunión (sin el profesor) se decidieron
algunos aspectos organizativos.}
\itemNvlUno{Se decidió la asignación definitva de los roles los miembros del grupo 
para el proyecto, tal y como se muestra en la parte superior del acta.}
\itemNvlUno{Se establecieron algunos aspectos de la política de reuniones:}
	\itemNvlDos{Los roles de director, secretario y planificador serán fijos para todas 
	las reuniones.}
	\itemNvlDos{Los roles de cronometrador y observador serán rotados entre los tres 
	miembros restantes del grupo, quedando así siempre un miembro libre de rol que nos 
	podrá venir bien para el caso de que algún miembro no pueda acudir a alguna reunión.}
	\itemNvlDos{Día y hora de las reuniones de seguimiento periódicas: Miércoles a las 20:00h}
\itemNvlUno{Se establecieron algunas políticas a seguir en situaciones conflictivas:}
	\itemNvlDos{Si algún miembro llega tarde a una reunión, dar comienzo a la misma 
	sin su presencia.}
	\itemNvlDos{La toma de decisiones serán conjunta. Se escucharán las argumentaciones 
	de cada miembro y se realizará una votación.}	
\itemNvlUno{Cuando comenzó la reunión, cada uno de los asistentes se presentó a la hora.}	
\itemNvlUno{Cada uno de los asistentes planteó brevemente su vida y experiencias en 
general, y lo relacionado al desarrollo de software en particular.}
\itemNvlUno{El profesor asistente planteó de manera detallada y general en que suele 
consistir el desarrollo de un proyecto software de pequeña escala y los aspectos 
importantes a considerar en el desarrollo de este proyecto.}
\itemNvlUno{El profesor asistente especificó la manera en que debiera presentarse 
la propuesta del proyecto software.}
\itemNvlUno{Se decidió fijar como cliente a una tienda de venta de 
microcontroladores, de tal forma que estuviese muy acotada la solución, y a
su vez se plantease como muy sencilla.}
\itemNvlUno{Se recordaron los requisitos solicitados de la aplicación por los
profesores de la asignatura.}
\itemNvlUno{Tras recalcar el profesor los requisitos no funcionales especificados,
se decidió que la aplicación fuera desarrollada como una aplicación Web utilizando 
para ello las tecnologías PHP, MySQL y CodeIgniter.}
\itemNvlUno{Se realizó una foto del equipo.}\\

\pagebreak
\hline % ------------------------------------------------------------------------
\cabeceraTabla{Decisiones tomadas}  \\
\hline % ------------------------------------------------------------------------

\itemNvlUno{Se ha decidido que el catálogo electrónico de la tienda fuese una 
sobre la venta de microcontroladores.}
\itemNvlUno{Se establecen como tecnologías para el desarrollo de la aplicación PHP,
MySQL y CodeIgniter.}\\

\hline % ------------------------------------------------------------------------
\cabeceraTabla{Temas pendientes} \\
\hline % ------------------------------------------------------------------------

\itemNvlUno{Rotular la fotografía del equipo para que los profesores de la asignatura
puedan así identificar mejor a los componentes del equipo.}\\

\cabeceraTabla{Próxima reunión prevista} \\
\hline % ------------------------------------------------------------------------
Jueves 06 de Marzo del 2014 a las 10:00 
\end{longtable}


\end{document}